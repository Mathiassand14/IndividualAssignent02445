\section{Part 1}

	\subsection{Paper Selection}
		Positive Paper: \emph{Automated detection of lung nodules and coronary artery calcium using artificial intelligence on low-dose CT scans for lung cancer screening: accuracy and prognostic value} \cite{paperX}.\\
		Negative Paper: \emph{Hospital quality classification based on quality indicator data during the COVID-19 pandemic} \cite{paperY}.

	\subsection{Positive Paper Evaluation}
		The positive paper presents a robust evaluation of an AI model for detecting lung nodules and coronary artery calcium.
		They use heart scaling to ensure consistent heart size in images, and the models are well-documented.
		Additionally, the model is validated by two expert radiologists, providing a high level of confidence in the
		results.
		Results are supported by confidence intervals and p-values, and the paper includes extensive
		statistical testing.
		They use multiple tests and test assumptions for these tests, enhancing the
		reliability of their findings.
		The correlation biplot is a useful visualization tool that helps to understand the relationship between
		model predictions and various attributes of the patients.
		The paper uses a train/test split for model
		evaluation.

	\subsection{Negative Paper Evaluation}
		The negative paper suffers from several critical issues that undermine its findings.
		The main problem is the bias introduced by inadequate data submissions from hospitals, leading to a
		non-representative dataset.
		The authors attempt to classify hospital quality during the COVID-19 pandemic.
		However, they do not account because these hospitals don't suddenly change their care quality.
		Then after the pandemic, they will find that all the hospitals are suddenly substantially better, which is
		not the case.
		The article uses data transformation techniques, such as averaging per year and collapsing months to a
		single value, thus removing all trends and seasonality in the data.
		In addition, the paper replaces NULL
		values with 0 this lowers the possible score and introduces further bias.
		The Neural Network (NN) and Decision Tree (DT) models are overfitted to the data, as they do not use early
		stopping or two-level cross-validation.
		The authors then conclude that the linear discriminant analysis is the best model, but this conclusion is
		flawed because the data is not representative of the population and the models are overfitted.
		When they report results, they do not provide confidence intervals or statistical significance for their
		results, without which it is impossible to assess the reliability of their findings.
		There is no standardization or normalization of the data, which helps some models to perform better.
		Finally, the paper lacks a sufficient description of the models used, making it difficult to replicate their findings.

	\subsection{Recommendations for Improvement of the Negative Paper}
		To improve the negative paper, the authors should address the following issues:


% …

		\begin{itemize}[topsep=2pt, itemsep=1pt, parsep=0pt, partopsep=0pt]
			\item Ensure that the dataset is representative of the population by addressing the bias introduced by inadequate data submissions from hospitals.
			\item Avoid replacing NULL values with 0, as this can introduce bias and lower the possible score.
			\item Use early stopping and two‐level cross‐validation to prevent overfitting of the models.
			\item Provide confidence intervals and statistical significance for the results to assess the reliability of the findings.
			\item Standardize and normalize the data to improve model performance.
			\item Include a more detailed description of the models used to allow for replication of the findings.
			\item Consider using a more appropriate problem formulation, such as predicting future hospital quality rather than classifying past data.
		\end{itemize}

		\begin{thebibliography}{9}

			\bibitem{paperX}
			Chamberlin, J., Kocher, M.~R., Waltz, J., Snoddy, M., Stringer, N.~F.~C., Stephenson, J.,
			Sahbaee, P., Sharma, P., Rapaka, S., Schoepf, U.~J., Abadia, A.~F., Sperl, J., Hoelzer, P.,
			Mercer, M., Somayaji, N., Aquino, G., \& Burt, J.~R. (2021).
			\emph{Automated detection of lung nodules and coronary artery calcium using artificial intelligence on low-dose CT scans for lung cancer screening: accuracy and prognostic value}. \textbf{BMC Medicine}.

			\bibitem{paperY}
			Nurhaida, I., Dhamanti, I., Ayumi, V., Yakub, F., \& Tjahjono, B. (2024).
			\emph{Hospital quality classification based on quality indicator data during the COVID-19 pandemic}.
			\textbf{International Journal of Electrical and Computer Engineering}


		\end{thebibliography}